%%%%
% MTecknology's Resume
%%%%
% Author: George Fam
% License: CC-BY-4
% - https://creativecommons.org/licenses/by/4.0/legalcode.txt
%%%%
\documentclass[letterpaper,10pt]{article}
\usepackage{mteck}
\usepackage{fancyhdr}

%===================%
% George Fam's Resume %
%===================%

%\numberedPages % NOTE: lastpage requires a second build
%\documentFooter{\tab \thepage \ / 2} % Does similar without using lastpage
\begin{document}
  %---------%
  % Heading %
  %---------%
  \documentTitle{George Fam}{
    % Web Version
    %\raisebox{-0.05\height} \faPhone\ [redacted - web copy] ~
    %\raisebox{-0.15\height} \faEnvelope\ [redacted - web copy] ~
    %%
    \href{tel:5146639709}{
      \raisebox{-0.05\height} \faPhone\ 514-663-9709} ~ | ~
    \href{mailto:george.fam@famcode.net}{
      \raisebox{-0.15\height} \faEnvelope\ george.fam@famcode.net} ~ | ~
    \href{https://www.linkedin.com/in/george-fam/}{
      \raisebox{-0.15\height} \faLinkedin\ linkedin.com/in/george-fam}
  }

  %---------%
  % Summary %
  %---------%

  \tinysection{Sommaire}
  Étudiant en informatique et génie logiciel à l’UQAM, passionné par le développement logiciel, l’administration des systèmes et l'enseignement des sciences informatiques. En tant qu’auxiliaire d’enseignement, j’accompagne les étudiants en laboratoire et lors de la correction d’évaluations, développant des compétences en pédagogie et vulgarisation technique. Mon expérience en pharmacie m’a également renforcé en communication et en résolution de problèmes.

  %-----------%
  % Education %
  %-----------%

  \section{Éducation}

  \headingBf{Baccalauréat, Informatique et Génie logiciel}{2023 -- En Cours} % Note: Adding year(s) exposes an implied age
  \headingIt{UQÀM | Université de Québec à Montréal}{}
  \vspace{10pt}
  \headingBf{Double DEC, Science et Sciences Sociales}{2023} % Note: Adding year(s) exposes an implied age
  \headingIt{CEGEP John Abbott}{}

  %--------%
  % Skills %
  %--------%

  \section{Compétences}

  \begin{multicols}{2}
    \begin{itemize}[itemsep=-2px, parsep=5pt, leftmargin=75pt]
      \item[\textbf{Langages}] Java, C,, Bash, HTML/CSS/JS
      \item[\textbf{Outils}] Git, Maven, Jira, Docker, Podman, SQLite
      \item[\textbf{Tests}] JUnit, BATS
      \item[\textbf{OS}]  Windows, Linux (Arch, Debian, Ubuntu)
      \item[\textbf{Méthodologies}]  Scrum, Kanban, Agile
      \item[\textbf{Aptitudes}]  Communication, Travail d'équipe, Résolution de conflits, Écoute active, Service client
      \item[\textbf{Pharmacie}]  RxPro, AssystRx,Ubiq, Empego, Magistrale, Commande de médicaments, Préparation des ordonnances
      \item[\textbf{Langues}]  Français, Anglais, Arabe
    \end{itemize}
  \end{multicols}

  %------------%
  % Experience %
  %------------%

  \section{Expérience}

  \headingBf{Auxiliaire d'Enseignement}{Oct. 2024 -- Présent}
  \headingIt{UQÀM | Université de Québec à Montréal}{}
  \begin{resume_list}
    \itemTitle{Démonstrateur de laboratoire}
    \item Offrir un encadrement pratique aux étudiants en expliquant les concepts et en répondant à leurs questions techniques
    \begin{itemize}
        \item INF2120 — Programmation II (Été 2025)
        \item INF1070 — Utilisation et administration des systèmes informatiques (Hiver 2025)
        \item INF2050 — Outils et pratiques de développement logiciel (Hiver 2025)
    \end{itemize}
  \vspace{3pt}
  
    \itemTitle{Correcteur}
    \item Assurer une correction rigoureuse et objective des examens en respectant les barèmes établis
    \begin{itemize}
        \item INF2050 — Outils et pratiques de développement logiciel (Automne 2024, Hiver 2025)
    \end{itemize}
  \end{resume_list}

  \headingBf{Assitant Technique en Pharmacie}{Sept. 2022 -- Présent}
  \headingIt{BeLocum}{Août 2023 -- Présent}
  \headingIt{Jean Coutu}{Sept. 2022 -- Fév. 2025}
  \begin{resume_list}
    \item Interpréter et saisir des ordonnances avec précision et efficacité à l'aide de RxPro, AssystRx et Ubiq
  \end{resume_list}

  \headingBf{Tuteur}{Oct. 2022 -- Juin 2023}
  \headingIt{Projet Communautaire Pierrefonds}{}
  \begin{resume_list}
    \item Concevoir et animer des séances de tutorat extrascolaires engageantes pour les étudiants
    \item Aider les élèves à faire leurs devoirs et à se préparer aux examens
    \item Coordonner des séances d'apprentissage en petits groupes et des jeux interactifs pour stimuler la motivation scolaire
    \item Développer et mettre en œuvre des exercices supplémentaires pour améliorer la compréhension des élèves
    \item Corriger des exercices pour permettre aux étudiants de comprendre leurs erreurs et de s'améliorer
  \end{resume_list}

  \headingBf{Assistant Gérant}{Juil. 2018 -- Juil. 2020}
  \headingIt{Couche-Tard}{}
  \begin{resume_list}
    \item Former les nouveaux membres de l'équipe sur le système de point de vente et le service à la clientèle
    \item Faire les rapports financiers et les dépôts bancaires
  \end{resume_list}

  \pagebreak

    %----------------------------%
  % Extracurricular Activities %
  %----------------------------%

  \section{Travaux Pratiques}

  \headingBf{INF2120 (Programmation II)}{Java, JUnit}
  \begin{resume_list}
    \item Conception d’algorithmes à partir d'une interface
    \vspace{2pt}
    \item Développement d’un émulateur pour un robot virtuel
    \vspace{2pt}
    \item Développement de fonctionnalités pour un jeu de type Tower Defence
    \vspace{2pt}
    \item Conception et gestion de structures de données telles que des ensembles disjoints pour modéliser des intervalles triés
    \vspace{2pt}
    \item Réalisation de tests unitaires pour assurer la robustesse et la fiabilité des fonctionnalités implémentées
    
  \end{resume_list}

  \headingBf{INF2050 (Outils et pratiques de développement logiciel)}{Java, JUnit, Maven}
  \begin{resume_list}
    \item Développement d’une application pour calculer la valeur foncière de terrains en fonction de données JSON d’entrée, avec génération de résultats JSON en sortie
    \vspace{2pt}
    \item Utilisation de Maven pour la gestion des dépendances, la compilation et la production de JAR exécutables
    \vspace{2pt}
    \item Écriture et exécution de tests unitaires automatisés à l'aide de Maven et JUnit
  \end{resume_list}
  
  \headingBf{INF3135 (Construction et maintenance de logiciels)}{C, Bash, BATS}
  \begin{resume_list}
    \item Développement du jeu vidéo \textit{Circles}, inspiré de \textit{Super Hexagon}, en utilisant la bibliothèque SDL2 pour la programmation graphique et l’animation
    \vspace{2pt}
    \item Utilisation de Bash et de BATS pour écrire et automatiser des tests
    \vspace{2pt}
    \item Gestion de projet en utilisant Git et une approche collaborative via des branches, avec des merge requests et un versionnement clair
  \end{resume_list}
  
 %----------------------------%
  % Volunteering %
  %----------------------------%

  \section{Implication Sociale}

  \headingBf{Tuteur (bénévole)}{2016 -- 2017 \& 2022}
  \headingIt{Projet Communautaire Pierrefonds}{Janv. 2022 -- Juin 2022}
  \headingIt{École Secondaire St-Georges}{Oct. 2016 -- Mars 2017}
  \begin{resume_list}
    \item Aider les élèves avec leurs devoirs
    \item Vulgariser des concepts de mathématiques et de sciences
  \end{resume_list}
  
  \headingBf{Préposé Bénévole aux Activités pour Patients}{Oct. 2016 -- Fév. 2017}
\headingIt{Fondation de l'Hôpital Sainte-Anne}{}
\begin{resume_list}
    \item Accompagner et assister les patients dans leurs routines et activités quotidiennes
    \item Participer à l'organisation d'activités récréatives, telles que des parties de Bingo
\end{resume_list}


\end{document}
