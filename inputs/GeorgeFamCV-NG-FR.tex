%%%%
% MTecknology's Resume
%%%%
% Author: George Fam
% License: CC-BY-4
% - https://creativecommons.org/licenses/by/4.0/legalcode.txt
%%%%
\documentclass[letterpaper,10pt]{article}
\usepackage{mteck}
\usepackage{fancyhdr}

%===================%
% George Fam's Resume %
%===================%

\begin{document}
%---------%
% Heading %
%---------%
\documentTitle{George Fam}{
    \href{tel:5146639709}{
    \raisebox{-0.05\height} \faPhone\ 514-663-9709} ~ | ~
    \href{mailto:fam.george@courrier.uqam.ca}{
    \raisebox{-0.15\height} \faEnvelope\ fam.george@courrier.uqam.ca} ~ | ~
    \href{https://www.linkedin.com/in/george-fam/}{
    \raisebox{-0.15\height} \faLinkedin\ linkedin.com/in/george-fam} ~ | ~
    \href{https://www.github.com/George-Fam}{
    \raisebox{-0.15\height} \faGithubSquare\ github.com/George-Fam}
}

%---------%
% Summary %
%---------%

\tinysection{Sommaire}
Étudiant en génie logiciel ayant un fort intérêt pour le développement backend, les systèmes infonuagiques et l’ingénierie full-stack. J’aime concevoir des applications fiables en utilisant Java, Spring Boot, Angular, Docker et Linux, et j’ai déployé des systèmes complets sur GCP dans le cadre de hackathons. Solide base en algorithmes, structures de données et programmation orientée objet, avec de l’expérience supplémentaire en C/C++ ainsi qu'en scripts sous Windows et Linux.
\newline

%---------%
% Achievements %
%---------%

\tinysection{Réalisations}
\textbf{2\textsuperscript{ème} Place - Hackathon Morgan Stanley Code to Give (2025)}   Crée \emph{Donate With Athena},une plateforme complète de dons et d’engagement pour l’organisme Shield of Athena lors d’un hackathon de cinq jours. Contribution au développement backend, frontend, au déploiement ainsi qu’à l’intégration de fonctionnalités IA dans des délais serrés.

%-----------%
% Education %
%-----------%

\section{Éducation}

\headingBf{Baccalauréat, Informatique et Génie logiciel}{2023 -- En Cours}
\headingIt{UQÀM | Université du Québec à Montréal}{}

%--------%
% Skills %
%--------%

\section{Compétences}

\begin{multicols}{2}
    \begin{itemize}[itemsep=-2px, parsep=5pt, leftmargin=75pt]
        \item[\textbf{Languages}] Java, C, C++, TypeScript, JavaScript, Bash, PowerShell
        \item[\textbf{Frontend}] HTML/CSS, Tailwind CSS, Angular
        \item[\textbf{Backend}] Spring Boot, Spring MVC, Spring Data JPA, REST APIs
        \item[\textbf{Databases}] MySQL, Oracle SQL	
        \item[\textbf{Testing}] JUnit, Mockito, BATS
            \columnbreak
        \item[\textbf{Fondements TI}] Structures de Données \& Algorithmes, OOP, Analyse de Complexité
        \item[\textbf{DevOps}] Docker, Docker Compose, Podman, Nginx, Git, Maven
        \item[\textbf{Cloud}] Google Cloud (Compute Engine, Réseautique, SSL, Déploiement)
        \item[\textbf{Languages}] Français, Anglais, Arabe
    \end{itemize}
\end{multicols}

%------------%
% Experience %
%------------%

\section{Expérience}

\headingBf{Auxiliaire d'Enseignement}{Oct. 2024 -- Présent}
\headingIt{UQÀM | Université de Québec à Montréal}{}
\begin{resume_list}
    \itemTitle{Animateur d’atelier Linux}
\item Accompagner les étudiants dans l’installation, la configuration et le dépannage de Linux.
\item Rédiger et maintir un guide complet d’installation Linux. {\small \href{https://github.com/George-Fam/LinuxInstallGuide}{\faGithubSquare\ George-Fam/LinuxInstallGuide}}
    \vspace{1pt}
    \itemTitle{Démonstrateur de laboratoire}
\item Offrir un soutien pratique en laboratoire en expliquant les concepts fondamentaux, en répondant aux questions techniques et en aidant au débogage et à la résolution de problèmes dans 13 laboratoires par cours.
\item Concevoir des supports visuels et des questionnaires afin d’améliorer l’apprentissage pour des groupes allant jusqu’à 50 étudiants. {\small\href{https://github.com/George-Fam/UQAM-Labos}{\faGithubSquare\ George-Fam/UQAM-Labos}}
    \vspace{1pt}
    \itemTitle{Correcteur}
\item Assurer une correction rigoureuse et objective des examens conformément aux directives académiques établies.
    \vspace{2pt}
\item Cours: Programmation II, Utilisation \& Admin. de Systèmes Linux, Outils et Pratiques de développement
\end{resume_list}

\headingBf{Assitant Technique en Pharmacie}{Sept. 2022 -- Fév. 2025}
\headingIt{Jean Coutu}{}
\begin{resume_list}
\item Offrir un service à la clientèle à haut volume et assurer la saisie de données médicales, en traitant plus de 500 ordonnances par jour avec rigueur, exactitude et confidentialité.
\item Trier et prioriser les demandes entrantes (nouvelles ordonnances, renouvellements, assurances, clarifications) afin de respecter les délais de service critiques.
\end{resume_list}

\headingBf{Tuteur}{Oct. 2022 -- Juin 2023}
\headingIt{Projet Communautaire Pierrefonds}{}
\begin{resume_list}
\item Aider les étudiants à faire leurs devoirs et à se préparer aux examens.
\item Organiser des séances d’apprentissage en petits groupes et des activités interactives afin de stimuler la motivation et la réussite scolaire.
\end{resume_list}

\headingBf{Assistant Gérant}{Juil. 2018 -- Juil. 2020}
\headingIt{Couche-Tard}{}
\begin{resume_list}
\item Former les nouveaux membres de l’équipe à l’utilisation du système de point de vente et aux standards de service à la clientèle.
\item Préparer les rapports financiers et effectuer les dépôts bancaires.
\end{resume_list}

\pagebreak

%----------------------------%
% Extracurricular Activities %
%----------------------------%

\section{Projects}

\headingBf{Donate With Athena {\small \href{https://github.com/Morgan-Stanley-CtG-Team-3/CodeToGiveTeam3Montreal}{\faGithubSquare\ Donate With Athena}}}{Spring Boot, Angular, MySQL, Docker, Nginx, GCP, Tailwind, n8n}
\begin{resume_list}
\item Développer une plateforme full-stack de dons, d’abonnements et d’engagement pour Shield of Athena lors du hackathon Code to Give de Morgan Stanley.
    \vspace{2pt}
\item Concevoir et implémenter des API REST (dons, événements, questionnaires, badges, abonnements) avec Spring Boot et MySQL (JPA/Hibernate).
    \vspace{2pt}
\item Contribuer au frontend Angular en intégrant des API REST et en participant à l’implémentation du support multilingue.
    \vspace{2pt}
\item Déployer la plateforme sur Google Cloud en utilisant Docker Compose, un proxy inverse Nginx et SSL.
    \vspace{2pt}
\item Intégrer des flux de travail IA (ChatBot, VoiceBot téléphonique, création et publication automatisée de contenu) avec n8n et Tidio pour améliorer l’assistance et l’engagement des donateurs.
    \vspace{2pt}
\item Centraliser le soutien aux donateurs, les dons récurrents, les questionnaires de sensibilisation, la progression de badges, les événements et les notifications automatisées.
\end{resume_list}

\headingBf{LaTeX CV \& Cover Letter Automation {\small\href{https://github.com/George-Fam/Resume}{\faGithubSquare\ George-Fam/Resume}}}{PowerShell, LaTeX}
\begin{resume_list}
\item Concevoir un système d’automatisation en PowerShell pour compiler des CV et lettres de motivation multilingues.
\item Ajouter la compilation par lot ou ciblée, l’intégration de signatures numériques et le suivi de versions.
\end{resume_list}

\headingBf{Toolbelt {\small \href{https://github.com/George-Fam/Toolbelt}{\faGithubSquare\ George-Fam/Toolbelt}}}{PowerShell}
\begin{resume_list}
\item Créer une collection d’utilitaires, incluant un bootstrapper Git, un visualiseur de backlog Plex et un script privilégié de suppression forcée.
\end{resume_list}

\headingBf{Implémentation d'Arbre AVL (INF3105: Structures de Données et Algorithmes)}{C++}
\begin{resume_list}
\item Implémenter un arbre AVL prenant en charge l’insertion, la suppression, la recherche et le rééquilibrage automatique; classé \#1 du groupe sur le projet.
    \vspace{2pt}
\item Analyser la complexité et valider la justesse à l’aide de tests structurés.
\end{resume_list}

\headingBf{Land Value Analyzer (INF2050: Outils et Pratiques de Développement)}{Java, JUnit, Maven}
\begin{resume_list}
\item Implémenter un calculateur de valeur immobilière basé sur JSON avec sortie structurée.
    \vspace{2pt}
\item Utiliser Maven pour gérer les dépendances, automatiser le build et exécuter les tests.
\end{resume_list}

\headingBf{Circles (INF3135: Construction et Maintenace de Logiciels)}{C, SDL2, Bash, BATS}
\begin{resume_list}
\item Développer un jeu arcade inspiré de \textit{Super Hexagon} en utilisant SDL2 pour les animations et les graphiques.
\end{resume_list}

\headingBf{Tower Defense Game (INF2120: Programmation II)}{Java, Swing, JUnit}
\begin{resume_list}
\item Ajouter des fonctionnalités de jeu et des composantes d’interface pour un jeu de type Tower Defense en Swing.
\end{resume_list}

\end{document}
