\begin{document}
\fontfamily{phv}\selectfont
\userInfo{Montréal, Québec}{(514) 663-9709}{george.fam@famcode.net}
\companyInfo{Orthogone}
\object{Candidature au stage de Concepteur logiciel}
\content{
  Je poursuis actuellement des études en informatique et génie logiciel à l’UQAM, et je souhaite contribuer aux projets d’Orthogone dans un environnement stimulant et innovant. Mon intérêt pour les systèmes logiciels fiables, mon expérience en encadrement technique, ainsi que les compétences que j’ai développées dans mes cours et projets me motivent à poser ma candidature pour ce stage. \\

Durant mon parcours, j’ai eu l’occasion d’accompagner les étudiants dans les cours INF2050 (Outils et pratiques de développement logiciel) et INF2120 (Programmation II), en les aidant à comprendre des concepts comme la programmation orientée objet, les tests unitaires (JUnit), et l’utilisation d’outils comme Git, Maven et Jenkins. J’ai aussi participé à la correction d’évaluations, ce qui m’a permis de développer un bon sens de rigueur et de clarté dans la communication technique — des compétences essentielles en développement collaboratif. Ces mêmes outils et méthodologies ont été mis en pratique dans mes projets académiques, où j’ai pu travailler sur des applications en Java et C, incluant des tests automatisés, des processus de revue de code (pull requests) et une gestion de projet structurée avec GitFlow.\\
\\
Je m’intéresse particulièrement aux environnements Linux et systèmes embarqués. Je suis curieux d’approfondir mes connaissances avec vous. Mon autonomie, ma capacité d’adaptation et mon intérêt pour la qualité du code font de moi un candidat motivé à apprendre et à contribuer activement aux projets de votre équipe.\\
}{Merci de l’attention portée à ma candidature. Ce serait un plaisir d’échanger avec vous au sujet de vos projets et de la manière dont je pourrais y contribuer.}
\signature
\includegraphics[height=3\baselineskip]{inputs/signature} \par
\end{document}